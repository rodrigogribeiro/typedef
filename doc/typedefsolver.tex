\documentclass[a4paper,8pt]{article}
\usepackage[utf8]{inputenc}
\usepackage[brazil]{babel}
\usepackage{amsmath,amsthm,thmtools}
\usepackage{amsfonts}
\usepackage{proof}
\usepackage{color}
\usepackage{amssymb}

\newcommand{\constr}[1]{\ensuremath{\langle\langle\,#1\,\rangle\rangle}}

\begin{document}

     \title{Reconstrução de tipos em C}

     \maketitle
     

     \section{Introdução}\label{intro}

     O objetivo deste texto é apresentar o desenvolvimento de um
     protótipo de ferramenta para inferir declarações de tipos não
     presentes em programas C. Neste documento descreveremos
     a sintaxe núcleo para programas C (Seção~\ref{corecsyn}), a
     sintaxe de restrições (Seção~\ref{constraintsyn}) e a descrição
     do resolvedor de restrições utilizado
     (Seção~\ref{constraintsolver}).

     \section{Sintaxe da Linguagem Fonte}\label{corecsyn}

     Nas figuras seguintes, apresentamos a sintaxe da linguagem núcleo
     considerada. Nesta figuras, utilizaremos as seguintes metavariáveis.

     \begin{figure}[h]
     \begin{tabular}{|l|l|}
       \hline
       Símbolo & Significado \\ \hline
		$l$ & {literal} \\
		$x$ & {variável}\\
		$f$  & {identificador de função} \\
		$\tau$ & {tipo}\\
		$\circ$ & {operador binário qualquer}\\
                $\rho : l \to \tau$ & {função para atribuir tipos a
                                      literais}\\
                $P$ & {programa} \\
                $D$ & {declaração} \\
                $C$ & {comando} \\
                $E$ & {expressão}\\
		\hline
		\end{tabular}
           \centering
           \caption{Metavariáveis}
           \label{figmetavar}
       \end{figure}

     Neste trabalho consideramos que um programa é uma lista de
     declarações que podem ser definições de tipos (typedef's) e
     funções. Uma definição de tipo é formada por um tipo e o novo 
     nome atribuído a este. Por sua vez, Funções são descritas por seu 
     nome, tipo de retorno, parâmetros e uma lista de comandos que 
     compõe o corpo desta. Abaixo apresentamos a sintaxe de
     declarações, em que elementos em fonte \texttt{typewriter}
     denotam terminais.
     
     \begin{figure}[h]
         \[
               \begin{array}{lcl}
                   P & ::=    & D^*\\
                   D & ::=   & \texttt{typedef }\tau\:\:x\\
                      & \mid & \tau\:f\:\texttt{(}ps\texttt{)} \texttt{\{}\mathbb{C}^*\texttt{\}} \\
                   ps & ::= & \tau\:x \,\mid\,(\texttt{,}ps)^*
               \end{array}
         \]
         \centering
         \caption{Sintaxe de programas}
         \label{figprogsyn}
     \end{figure}

     A sintaxe de tipos é bastante direta. Representaremos tipos de
     funções usando o construtor de tipos $\to$, de maneira similar ao
     utilizado em linguagens funcionais como Haskell. 
     Outro ponto é que não consideramos
     como tipo da linguagem arranjos visto que estes podem ser
     representados como ponteiros.

     \begin{figure}[h]
       \[
         \begin{array}{lcll}
           \tau & ::= & \textbf{B} & \{\text{construtores de tipos:
                                     int, bool, etc.}\}\\
                & \mid & \star\, \tau & \{\text{ponteiros}\} \\
                & \mid & \{x_i : \tau_i\}^{i=1..n} & \{\text{registros}\} \\
                & \mid & \tau^* \to \tau & \{\text{tipos de funções}\}\\
         \end{array}
       \]
       \centering
       \caption{Sintaxe de tipos}
       \label{figtypesyn}
     \end{figure}
     
     Existem 10 tipos de comandos na linguagem considerada: Chamadas de
     função, retorno de função, definição de variável, atribuição de
     variável, atribuição a um item de um arranjo, atribuição a um
     ponteiro, atribuição a um campo (de registro), atribuição a um
     campo em um ponteiro para registro, comandos de repetição e condicional.

     \begin{figure}[h]
       \[
         \begin{array}{lcll}
           \mathbb{C} & ::=   & \tau\:x\:=\:e & \{\text{var. def.}\} \\
             & \mid & x\:=\:e & \{\text{atrib. var.}\} \\
             & \mid & \star\,x\:=\:e & \{\text{atrib. pont.}\} \\
             & \mid & x[e]\:=\:e &\{\text{atrib. arranjo}\} \\
             & \mid & x.x'\: = \: e & \{\text{atrib. campo}\} \\
             & \mid & x\to x\: = \: e & \{\text{atrib. campo pont.}\}\\
             & \mid & f(e^*) & \{\text{chamada de função}\} \\
             & \mid & \texttt{return }e & \{\text{retorno}\} \\
             & \mid & \texttt{while (}e \texttt{)\{} \mathbb{C}^*
                      \texttt{\}} & \{\text{while}\} \\
             & \mid & \texttt{if (} e \texttt{) \{}\mathbb{C}^*
                      \texttt{\}} & \{\text{if}\} \\
         \end{array} \]
       \centering
       \caption{Sintaxe de comandos}
       \label{figcmdsyn}
    \end{figure}

    A sintaxe de expressões é apresentada na figura seguinte. A
    linguagem núcleo considerada consiste de literais, variáveis,
    acesso a campo, acesso a arranjo, coerção, expressões contendo
    operadores binários, chamadas de função e outras expressões
    relacionadas ao uso de ponteiros.

    \begin{figure}[h]
       \[
         \begin{array}{lcll}
           e & ::= & l & \{\text{literal}\} \\
             & \mid & x & \{\text{variável}\} \\
             & \mid & e . x & \{\text{acesso a campo}\}\\
             & \mid & e\,[e_1] & \{\text{accesso arranjo}\} \\
             & \mid & (\tau)\,e & \{\text{casting}\} \\
             &\mid & e\circ e' & \{\text{bin op}\} \\
             & \mid & \&\,e & \{\text{address}\} \\
             & \mid & \star\,e & \{\text{deref. ponteiro}\}\\
             & \mid & e \to x & \{\text{ref. campo ponteiro}\}\\
             & \mid & f(e_i)^{i=0..n} & \{\text{chamada função}\}\\
         \end{array} \]
         \centering
         \caption{Sintaxe de expressões}
         \label{exprsyn}
    \end{figure}



     \section{Sintaxe da Linguagem de Restrições}\label{constraintsyn}

     Antes de apresentarmos a sintaxe de restrições, mostraremos sua
     interpretação intuitiva. Esta é apresentada abaixo.

       \begin{itemize}
           \item Existência de definição de tipo: $typedef\:\tau\:as\:$.
           \item Existência de declaração de variável:
             $def\:x\,:\,\tau \text{ in } \mathbb{K}$.
           \item Igualdade de tipos: $\tau \equiv \tau'$.
	   \item Existência de campos: $has(\tau,x:\tau')$ --- o campo $x'
             : \tau'$ é parte do tipo $\tau$.
           \item Nova variável de tipo: $\exists \,\alpha. \mathbb{K}$
       \end{itemize}

     A sintaxe de restrições é apresentada a seguir. 

     \begin{figure}[h]
        \[
             \begin{array}{lcl}
                 \mathbb{K} & ::= & \tau \equiv \tau \\
                                    & \mid & typedef\:\tau \\
                                    & \mid & has(\tau, x' : \tau') \\
                                    & \mid & def\:x:\tau\:\text{ in
                                             }\mathbb{K}\\
                                    & \mid &
                                             \exists\,\alpha.\mathbb{K}\\
                                    & \mid & \mathbb{K}\,\land\,\mathbb{K}
             \end{array}
        \]
        \centering
        \caption{Sintaxe de restrições}
        \label{figconstrsyn}
      \end{figure}

     \section{Semântica de Restrições}\label{semantics}

     A linguagem de restrições é uma ``interface'' entre a sintaxe
     concreta da linguagem C e o resolvedor de restrições. Como
     restrições são expressas como fórmulas da lógica extraídas do 
     programa analisado, o próprio programa constitui um modelo 
     para suas restrições.

     Assim como fórmulas da lógica de primeira ordem, restrições são
     interpretadas em um universo de Herbrand. A interpretação de
     constraints é dada por um julgamento 
     $\phi ; \varphi ; \Theta \vdash \mathbb{K}$ em que $\phi$ é uma
     função finita que associa variáveis (de tipos) a tipos, $\varphi$
     associa variáveis de programa a tipos e $\Theta$ é uma função
     que associa nomes de tipos a sua definição.

     \[
          \begin{array}{cc}
      \infer[_{(CEmpty)}]{\phi ; \varphi ; \Theta \vdash true}{} & 
      \infer[_{(CEq)}]{\phi ; \varphi ; \Theta \vdash \tau_1 =
                                                                   \tau_2}{\phi(\tau_1)
                                                                   =
                                                                   \phi(\tau_2)}\\
        \\ 
       \infer[_{(CConj)}]{\phi ; \varphi ; \Theta \vdash K_1 \land K_2}
                            {\phi ; \varphi ; \Theta \vdash K_1 & 
                             \phi ; \varphi ; \Theta \vdash K_2 } &
        \infer[_{(CExists)}]{\phi ; \varphi ; \Theta \vdash K}
                                {\phi [\alpha \mapsto \textbf{t}] ; \varphi ; \Theta \vdash
                                                                    \exists
                                                                    \alpha
                                                                    . K
                                                                    }
            \\                                 
\\

\infer[_{(CVar)}]{\phi ; \varphi ; \Theta \vdash x = \tau}
                     {\varphi(x) = \phi(\tau) } &
\infer[_{(CDef)}]{\phi ; \varphi ; \Theta \vdash def\:x : \tau\:in\:
                                                  K}
                      {\phi ; \varphi [x \mapsto \phi(\tau) ]; \Theta
                                                  \vdash K} \\
\\
\infer[_{(CHas)}]{\phi ; \varphi ; \Theta \vdash has(\tau, x : \tau')}
                      {x : \tau'  \in fields(\Theta(\tau))} &
\infer[_{(CTyDef)}]{\phi ; \varphi ; \Theta \vdash typedef\: \tau\: as\:
                                                              \tau'}
                        {\phi ; \varphi ; \Theta \vdash \Theta(\phi(\tau)) =
                                                              \phi(\tau')}                                      
                                               
            
          \end{array}
     \]

      
     \section{Coletando Restrições}\label{constrcollect}

     O processo de geração de restrições consiste em percorrer a
     sintaxe do programa para construção de uma fórmula que, se
     satisfazível, seu modelo é dado pelas declarações de tipo
     no programa fornecido como entrada. 

     É importante notar que é responsabilidade
     do processo de geração a implementação da função que calcula o
     tipo de literais ($\rho$).


     \begin{figure}[h]
       \[\small{
            \begin{array}{lcl}
                \constr{l : \tau} & = & \rho(l) \equiv \tau\\
                \constr{x : \tau} & = & x \equiv \tau \\
                \constr{\circ : \tau} & = & \circ \equiv \tau \\
                \constr{f : \tau^{1..n}\to \tau'} & = & f \equiv
                                                        \tau^{1..n}\to
                                                        \tau' \\
                \constr{e.x : \tau} & = & \exists
                                          \alpha_1\,\alpha_2. \constr{e
                                          : \alpha_1} \land \constr{x
                                          : \alpha_2} \land \\
                          & & has(\alpha_1,x:\alpha_2) \land \tau
                              \equiv \alpha_2\\
                \constr{e[e_1] : \tau} & = & \exists
                                         \alpha_1\alpha_2\alpha_3. \constr{e:\alpha_1}
                                             \land\constr{e_1: \alpha_2}\land\\
                         & & \alpha_1\equiv\star\,\alpha_3 \land \tau
                             \equiv\alpha_3 \land \alpha_2 = int\\
                \constr{(\tau')\,e:\tau} & = & \constr{e : \tau} \land
                                               \tau \equiv \tau'\\
                \constr{f(e^{i=1..n}) : \tau} & = & \exists
                                                    \alpha^{i=1..n}. \bigwedge_{i=1..n}\constr{e_i:\alpha_i}
                                                    \land \constr{f :
                                                    \alpha^{i=1..n}
                                                    \to \tau} \\
                     \constr{e \circ e' : \tau} & = & \exists
                                                      \alpha_1\alpha_2.\constr{e
                                                      :\alpha_1} \land
                                                      \constr{e' :
                                                      \alpha_2} \land
                                                      \constr{\circ :
                                                      \alpha_1\to\alpha_2\to\tau}\\
                     \constr{\&\,e : \tau} & = & \exists \alpha\,
                                                 \alpha'. \constr{e :
                                                 \alpha'} \land
                                                 \alpha \equiv \star \alpha'
                                                 \land \tau = \alpha\\
                     \constr{\star\,e : \tau} & = & \exists
                                                    \alpha. \constr{e
                                                    : \alpha} \land
                                                    \alpha = \star \tau\\
                     \constr{e\to x : \tau} & = & \exists
                                                  \alpha_1\alpha_2\alpha_3. \constr{e
                                                  :\alpha_1} \land
                                                  \constr{x :
                                                  \alpha_3} \land
                                                  \alpha_1 = \star\alpha_2
                                                  \land\\
                          & & has(\alpha,x : \alpha_3) \land \tau
                              \equiv \alpha_3\\
            \end{array}}
       \]
       \centering
       \caption{Gerador de restrições --- parte 1}
       \label{figgen}
     \end{figure}


     \begin{figure}[h]
       \[
          \small{
           \begin{array}{lcl}
               \constr{\emptyset, \tau} & = & true \\
               \constr{\tau'\:x := e; S, \tau} & = & \exists \alpha. \constr{e : \alpha} \land
                                              typedef\:\tau' \land
                                              def\,x:\tau\,in\,\constr{S
                                                     , \tau}
             \land \tau \equiv \alpha\\
               \constr{x := e ; S, \tau} & = & \exists \alpha. \constr{x : \alpha}\land\constr{e : \alpha} \land
                                                      \constr{S , \tau}\\ 
               \constr{\star\,x := e ; S, \tau} & = & \exists \alpha \alpha'. \constr{x : \alpha'}\land\constr{e : \alpha} \land
                                                      \alpha' \equiv
                                                \star
                                                      \alpha\land\constr{S
                                                      , \tau}\\ 
               \constr{x[e'] := e ; S , \tau} & = &
                                             \exists\alpha_1\alpha_2\alpha_3.\constr{e'
                                             : \alpha_1}\land \alpha_1
                                             = int \land \constr{e :
                                             \alpha_2} \land \\ 
                       & & \constr{x : \alpha_3} \land
                           \alpha_3 \equiv \star\,\alpha_2 \land
                           \constr{S , \tau}\\
               \constr{f(e^{i=1..n}) ; S , \tau} & = & \exists
                                            \alpha_{i=1..n +1}.\bigwedge_{i
                                            = 1..n}\constr{e_i :
                                            \alpha_i} \land \constr{f
                                            : \alpha_{i = 0.. n} \to
                                            \alpha_{n + 1}} \land
                                                \constr{S , \tau}\\
               \constr{\texttt{while (} e
             \texttt{)\{}C\texttt{\}} ; S , \tau} &= & \constr{e : bool}
                                                     \land
                                                     \constr{C
                                                            , \tau}
                                                                \land \constr{S,\tau}\\
             \constr{\texttt{if (} e \texttt{)\{} C
             \texttt{\}} ; S , \tau} & = & \constr{e : bool}
                                                     \land
                                                     \constr{C
                                       , \tau} \land \constr{S,\tau}\\
             \constr{\texttt{return }e, \tau} & = & \constr{e : \tau}
           \end{array}}
       \]
       \caption{Gerador de restrições --- parte 2}
     \end{figure}

     \begin{figure}[h]
       \[
         \small{
           \begin{array}{lcl}
               \constr{\emptyset} & = & true \\
               \constr{\texttt{typedef }\tau\:\:x\,,D} & = &
                                                             typedef\:x\land
                                                             x = \tau
                                                             \land \constr{D}\\
               \constr{\tau\:f\:\texttt{(}(\tau\:x)^{i=1..n}\texttt{)}
             \texttt{\{}C^*\texttt{\}} , D} & = & def\,\,f :
                                                       \tau_{i =
                                                       1.. n}\to \tau
                                                       \:in\: 
                                                       (def\,\,x_i :
                                                       \tau_i)_{i
                                                       = 1..n}\: in\:
                                                       \constr{C,\tau}
                                                       \land \constr{D}
           \end{array}
         }
       \]
       \caption{Gerador de restrições --- parte 3}
     \end{figure}
     
     \section{Resolvedor de Restrições}\label{constraintsolver}

     O resolvedor é apenas uma pequena modificação de um algoritmo de
     unificação. Este pode ser resumido nos seguintes passos:
     \begin{enumerate}
          \item Coletar definições, restrições de campos e inserir
            variáveis ``fresh'' no lugar de variáveis existencialmente
            quantificadas.
          \item Expandir definições nas restrições.
          \item Resolver restrições de igualdade usando unificação.
          \item Aplicar a substituição gerada pela unificação nas
            restrições de registros.
          \item Gerar definições de registros com base nas restrições
            de campos.
     \end{enumerate}

%      \section{Exemplos}

%      Nesta seção apresento exemplos enviados pelo Fernando e as
%      suas respectivas restrições.

%      \subsection{Exemplo 1 --- arquivo T0.c}

%      Código C.

%      \begin{verbatim}
% T foo (int i) {
%    return i + 1;
% }

%   \end{verbatim}

%      Restrição

%      \begin{verbatim}
% isdef T , 
%    (def foo = (int,T) in 
%       def i = int in 
%           exists x_1 . exists x_2 . i = x_1 , (x_2 = int , + =
%    (x_1,x_2,T)))   
%    \end{verbatim}

%      \subsection{Exemplo 2 --- arquivo T3.c}

%      Código C.

%      \begin{verbatim}
% int main (){
%    T t;
%    t.i = 0;
%    t.next = &t;
% }
%   \end{verbatim}

%      Restrição

%      \begin{verbatim}
% def main = (int) in 
%    isdef T , def t = T in , 
%       (exists x_1 . x_1 = int , (i = x_1 , has(t ,x_1 i)) , 
%          (exists x_2 . 
%               exists x_3 . t = x_3 , x_2 = x_3 * , 
%                   (next = x_2 , has(t ,x_2 next))))
%    \end{verbatim}

\end{document}
