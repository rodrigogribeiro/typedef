\documentclass[a4paper,8pt]{article}
\usepackage[utf8]{inputenc}
\usepackage[brazil]{babel}
\usepackage{amsmath,amsthm,thmtools}
\usepackage{amsfonts}
\usepackage{proof}
\usepackage{color}
\usepackage{amssymb}

\newcommand{\constr}[1]{\ensuremath{\langle\langle\,#1\,\rangle\rangle}}

\begin{document}

     \title{Reconstrução de tipos em C}

     \maketitle
     

     \section{Introdução}\label{intro}

     O objetivo deste texto é apresentar o desenvolvimento de um
     protótipo de ferramenta para inferir declarações de tipos não
     presentes em programas C. Neste documento descreveremos
     a sintaxe núcleo para programas C (Seção~\ref{corecsyn}), a
     sintaxe de restrições (Seção~\ref{constraintsyn}) e a descrição
     do resolvedor de restrições utilizado
     (Seção~\ref{constraintsolver}).

     \section{Sintaxe da Linguagem Fonte}\label{corecsyn}

     Nas figuras seguintes, apresentamos a sintaxe da linguagem núcleo
     considerada. Nesta figuras, utilizaremos as seguintes metavariáveis.

     \begin{figure}[h]
     \begin{tabular}{|l|l|}
       \hline
       Símbolo & Significado \\ \hline
		$l$ & {literal} \\
		$x$ & {variável}\\
		$f$  & {identificador de função} \\
		$\tau$ & {tipo}\\
		$\circ$ & {operador binário qualquer}\\
                $\rho : l \to \tau$ & {função para atribuir tipos a
                                      literais}\\
                $P$ & {programa} \\
                $D$ & {declaração} \\
                $C$ & {comando} \\
                $E$ & {expressão}\\
		\hline
		\end{tabular}
           \centering
           \caption{Metavariáveis}
           \label{figmetavar}
       \end{figure}

     Neste trabalho consideramos que um programa é uma lista de
     declarações que podem ser definições de tipos (typedef's) e
     funções. Uma definição de tipo é formada por um tipo e o novo 
     nome atribuído a este. Por sua vez, Funções são descritas por seu 
     nome, tipo de retorno, parâmetros e uma lista de comandos que 
     compõe o corpo desta. Abaixo apresentamos a sintaxe de
     declarações, em que elementos em fonte \texttt{typewriter}
     denotam terminais.
     
     \begin{figure}[h]
         \[
               \begin{array}{lcl}
                   P & ::=    & D^*\\
                   D & ::=   & \texttt{typedef }\tau\:\:x\\
                      & \mid & \tau\:f\:\texttt{(}ps\texttt{)} \texttt{\{}\mathbb{C}^*\texttt{\}} \\
                   ps & ::= & \tau\:x \,\mid\,(\texttt{,}ps)^*
               \end{array}
         \]
         \centering
         \caption{Sintaxe de programas}
         \label{figprogsyn}
     \end{figure}

     A sintaxe de tipos é bastante direta. A única peculiaridade é a
     sintaxe de tipos para funções que consistem de uma lista de tipos
     dos parâmetros (separados por vírgulas) em que o último elemento
     é o tipo de retorno desta. Outro ponto é que não consideramos
     como tipo da linguagem arranjos visto que estes podem ser
     representados como ponteiros.

     \begin{figure}[h]
       \[
         \begin{array}{lcll}
           \tau & ::= & \textbf{B} & \{\text{tipos básicos: void, int,
                                     etc.}\}\\
                & \mid & \star\, \tau & \{\text{ponteiros}\} \\
                & \mid & \{x_i : \tau_i\}^{i=1..n} & \{\text{registros}\} \\
                & \mid & \texttt{(}\tau^*, \tau\texttt{)} & \{\text{tipos de funções}\}\\
         \end{array}
       \]
       \centering
       \caption{Sintaxe de tipos}
       \label{figtypesyn}
     \end{figure}
     
     Existem 8 tipos de comandos na linguagem considerada: Chamadas de
     função, retorno de função, definição de variável, atribuição de
     variável, atribuição a um item de um arranjo, atribuição a um
     ponteiro, atribuição a um campo (de registro) e atribuição a um
     campo em um ponteiro para registro.

     \begin{figure}[h]
       \[
         \begin{array}{lcll}
           \mathbb{C} & ::=   & \tau\:x\:=\:e & \{\text{var. def.}\} \\
             & \mid & x\:=\:e & \{\text{atrib. var.}\} \\
             & \mid & \star\,x\:=\:e & \{\text{atrib. pont.}\} \\
             & \mid & x[e]\:=\:e &\{\text{atrib. arranjo}\} \\
             & \mid & x.x'\: = \: e & \{\text{atrib. campo}\} \\
             & \mid & x\to x\: = \: e & \{\text{atrib. campo pont.}\}\\
             & \mid & f(e^*) & \{\text{chamada de função}\} \\
             & \mid & \texttt{return }e & \{\text{retorno}\}
         \end{array} \]
       \centering
       \caption{Sintaxe de comandos}
       \label{figcmdsyn}
    \end{figure}

    A sintaxe de expressões é apresentada na figura seguinte. A
    linguagem núcleo considerada consiste de literais, variáveis,
    acesso a campo, acesso a arranjo, coerção, expressões contendo
    operadores binários, chamadas de função e outras expressões
    relacionadas ao uso de ponteiros.

    \begin{figure}[h]
       \[
         \begin{array}{lcll}
           e & ::= & l & \{\text{literal}\} \\
             & \mid & x & \{\text{variável}\} \\
             & \mid & e . x & \{\text{acesso a campo}\}\\
             & \mid & e\,[e_1] & \{\text{accesso arranjo}\} \\
             & \mid & (\tau)\,e & \{\text{casting}\} \\
             &\mid & e\circ e' & \{\text{bin op}\} \\
             & \mid & \&\,e & \{\text{address}\} \\
             & \mid & \star\,e & \{\text{deref. ponteiro}\}\\
             & \mid & e \to x & \{\text{ref. campo ponteiro}\}\\
             & \mid & f(e_i)^{i=0..n} & \{\text{chamada função}\}\\
         \end{array} \]
         \centering
         \caption{Sintaxe de expressões}
         \label{exprsyn}
    \end{figure}



     \section{Sintaxe da Linguagem de Restrições}\label{constraintsyn}

     Antes de apresentarmos a sintaxe de restrições, mostraremos sua
     interpretação intuitiva. Esta é apresentada abaixo.

       \begin{itemize}
           \item Existência de definição de tipo: $isdef\:\tau$.
           \item Existência de declaração de variável: $def\:x\,:\,\tau$.
           \item Igualdade de tipos: $\tau \equiv \tau'$.
	   \item Existencia de campos: $has(x,x:\tau')$.
           \item Nova variável de tipo: $\exists \,\alpha. \mathbb{C}$
       \end{itemize}

      
     \section{Coletando Restrições}\label{constrcollect}


     \begin{figure}[h]
       \[\small{
            \begin{array}{lcl}
                \constr{l : \tau} & = & \rho(l) \equiv \tau\\
                \constr{x : \tau} & = & x \equiv \tau \\
                \constr{\circ : \tau} & = & \circ \equiv \tau \\
                \constr{f : \tau^{0..n}\to \tau'} & = & f \equiv
                                                        \tau^{0..n}\to
                                                        \tau' \\
                \constr{e.x : \tau} & = & \exists
                                          \alpha_1\,\alpha_2. \constr{e
                                          : \alpha_1} \land \constr{x
                                          : \alpha_2} \land \\
                          & & has(\alpha_1,x:\alpha_2) \land \tau
                              \equiv \alpha_2\\
                \constr{e[e_1] : \tau} & = & \exists
                                         \alpha_1\alpha_2\alpha_3. \constr{e:\alpha_1}
                                             \land\constr{e_1: int}\land\\
                         & & \alpha_1\equiv\star\,\alpha_3 \land \tau
                             \equiv\alpha_3\\
                \constr{(\tau')\,e:\tau} & = & \constr{e : \tau} \land
                                               \tau \equiv \tau'\\
                \constr{f(e^{i=0..n}) : \tau} & = & \exists
                                                    \alpha^{i=0..n}. \bigwedge_{i=0..n}\constr{e_i:\alpha_i}
                                                    \land \constr{f :
                                                    \alpha^{i=0..n}
                                                    \to \tau} \\
                     \constr{e \circ e' : \tau} & = & \exists
                                                      \alpha_1\alpha_2.\constr{e
                                                      :\alpha_1} \land
                                                      \constr{e' :
                                                      \alpha_2} \land
                                                      \constr{\circ :
                                                      \alpha_1\to\alpha_2\to\tau}\\
                     \constr{\&\,e : \tau} & = & \exists \alpha\,
                                                 \alpha'. \constr{e :
                                                 \alpha'} \land
                                                 \alpha \equiv \star \alpha'
                                                 \land \tau = \alpha\\
                     \constr{\star\,e : \tau} & = & \exists
                                                    \alpha. \constr{e
                                                    : \alpha} \land
                                                    \alpha = \star \tau\\
                     \constr{e\to x : \tau} & = & \exists
                                                  \alpha_1\alpha_2\alpha_3. \constr{e
                                                  :\alpha_1} \land
                                                  \constr{x :
                                                  \alpha_3} \land
                                                  \alpha_1 = \star\alpha_2
                                                  \land\\
                          & & has(\alpha,x : \alpha_3) \land \tau
                              \equiv \alpha_3\\
            \end{array}}
       \]
       \centering
       \caption{Gerador de restrições}
       \label{figgen}
     \end{figure}

     \section{Resolvedor de Restrições}\label{constraintsolver}

\end{document}
